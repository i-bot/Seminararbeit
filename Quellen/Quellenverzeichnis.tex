\begin{thebibliography}{99}

\bibitem{schallplatte1}{UNI-Protokolle: Schallplatte, zuletzt aufgerufen am
01.07.2015 \newline
\url{http://www.uni-protokolle.de/Lexikon/Schallplatte.html}}

\bibitem{kassette1}{Tonaufzeichnung damals \& heute: Compact-Cassette: Ein
Renner vor allem bei Kindern, zuletzt aufgerufen am 01.07.2015 \newline
\url{http://www.tonaufzeichnung.de/medien/compactcassette}}

\bibitem{kassette2}{UNI-Protokolle: Tonband, zuletzt aufgerufen am 01.07.2015
\newline \url{http://www.uni-protokolle.de/Lexikon/Tonband.html}}

\bibitem{cd_durchbruch}{Bundesverband Musikindustrie: Musikindustrie in Zahlen,
S. 7, zuletzt aufgerufen am 01.07.2015 \newline
\url{http://www.musikindustrie.de/uploads/media/140325\_BVMI\_2013\_Jahrbuch\_ePaper\_V02.pdf}}

\bibitem{schellack}{VERBRAUCHER INITIATIVE e.V. (Bundesverband): Informationen
zu Lebensmittelzusatzstoffen, zuletzt aufgerufen am 23.07.2015 \newline
\url{http://www.zusatzstoffe-online.de/zusatzstoffe/274.e904_schellack.html}}

\bibitem{cds}{Schouhammer Immink, Kees A.: The Compact Disc Story* in Journal of
the Audio Engineering Society (Vol. 46, No. 5, 1998 May)}

\bibitem{cuz}{Roth, Klaus: CD, DVD0 \& Co.: Die Chemie der schillernden
Scheiben, in: Chemie in unserer Zeit (41/2007), S. 334-336}

\bibitem{cd1}{CCInfo: Compact-Disc, zuletzt aufgerufen am 03.08.2015 \newline
\url{http://www.elektronikinfo.de/audio/cd.htm}}

\bibitem{cfcd}{Didaktik der Chemie / Universität Bayreuth: Die Chemie und
Funktionsweise der CD und DVD, zuletzt aufgerufen am 07.08.2015 \newline
\url{http://daten.didaktikchemie.uni-bayreuth.de/umat/cd_dvd/cd_dvd.htm}}

\bibitem{cdp}{Asshoff, Jörg: Optische Datenspeicher - Der CD-Player \newline
\url{http://www.muenster.de/~asshoff/physik/cd/cdplayer.htm}}

\end{thebibliography}
