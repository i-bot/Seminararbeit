\begin{thebibliography}{99}

\bibitem{schallplatte1}{avacom: Museum für Energiegeschichte(n): 78, 45, 33 -
vom sanften Ton zum starken Sound. Die Schallplatte begeistert die Welt \newline
\url{www.energiegeschichten.de/cps/rde/xbcr/avacon-museum/Begleitheft_78_45_33.pdf}
(zuletzt aufgerufen am 02.11.2015)}

\bibitem{kassette}{Tonaufzeichnung damals \& heute: Compact-Cassette: Ein Renner
vor allem bei Kindern \newline
\url{http://www.tonaufzeichnung.de/medien/compactcassette} (zuletzt aufgerufen
am 01.07.2015)}

\bibitem{tonband}{DUDEN: Tonband \newline
\url{http://www.duden.de/rechtschreibung/Tonband} (zuletzt aufgerufen am
\newline 02.11.2015)}

\bibitem{schellack}{VERBRAUCHER INITIATIVE e.V. (Bundesverband): Informationen
zu Lebensmittelzusatzstoffen \newline
\url{http://www.zusatzstoffe-online.de/zusatzstoffe/274.e904_schellack.html}
(zuletzt aufgerufen am 23.07.2015)}

\bibitem{cds}{Schouhammer Immink, Kees A.: The Compact Disc Story*, in: Journal
of the Audio Engineering Society (Vol. 46, No. 5, 1998 May) \newline
\url{info.biz.hr/typo3/typo3_01/dummy-3.8.0/uploads/media/CD_TEH_Prilog_1_K.A.Imminkov_gorvor_pred_AES_om.pdf}
(zuletzt aufgerufen am 02.11.2015)}

\bibitem{cuz}{Roth, Klaus: CD, DVD \& Co.: Die Chemie der schillernden
Scheiben, in: Chemie in unserer Zeit (41/2007), S. 334-336}

\bibitem{cdp}{Asshoff, Jörg: Optische Datenspeicher - Der CD-Player \newline
\url{http://www.muenster.de/~asshoff/physik/cd/cdplayer.htm} (zuletzt aufgerufen
am 28.02.15)}

\bibitem{cdpf}{Hochschule Pforzheim: Spritzgießen von thermoplastischen
Kunststoffen \newline
\url{https://www.hs-pforzheim.de/De-de/Technik/Maschinenbau/laborbereiche/kunstofftechnik/verarb_kunsttk/Seiten/Spritzgiessen.aspx}
(zuletzt aufgerufen am 11.08.2015)}

\bibitem{cuzpc}{Hübner, Karl: Polycarbonate, in: Chemie in unserer Zeit
(37/2003), S. 366-368}

\bibitem{cuzpc2}{Roth, Klaus: CD, DVD \& Co.: Die Chemie der schillernden
Scheiben, in: Chemie in unserer Zeit (41/2007), S. 338}

\bibitem{cfcd}{Universität Bayreuth: Didaktik der Chemie: Die Chemie und
Funktionsweise der CD und DVD \newline
\url{http://daten.didaktikchemie.uni-bayreuth.de/umat/cd_dvd/cd_dvd.htm}
(zuletzt aufgerufen am 07.08.2015)}

\bibitem{cuzpe}{Hübner, Karl: Polycarbonate, in: Chemie in unserer Zeit
(37/2003), S. 367}

\bibitem{pop}{Patent DE60114468T2 20.07.2006: OPTIMIERUNG DER HERSTELLUNG VON
POLYCARBONAT DURCH UMESTERUNG \newline
\url{http://www.patent-de.com/20060720/DE60114468T2.html} (zuletzt aufgerufen am
22.08.2015)}

\bibitem{garoo}{ChemgaPedia: Polycarbonate: Technische Herstellung der
Polycarbonate II \newline
\url{http://www.chemgapedia.de/vsengine/vlu/vsc/de/ch/9/mac/stufen/polykondensation/polyester/polycarbonat.vlu/Page/vsc/de/ch/9/mac/stufen/polykondensation/polyester/polycarbgrenz.vscml.html}
(zuletzt aufgerufen am 12.09.2015)}

\bibitem{cdversuch}{Wikibooks: Organische Chemie für Schüler/ Kunststoffe:
Kunststoffrecycling: CD Recycling \newline
\url{https://de.wikibooks.org/wiki/Organische_Chemie_f%C3%BCr_Sch%C3%BCler/_Kunststoffe#a.29_CD_Recycling}
(zuletzt aufgerufen am 02.11.2015)}

\bibitem{pofacgif}{Technische Hochschule Nürnberg: POF Application Center:
GI-POF aus Fluorpolymer \newline
\url{http://www.pofac.info/startseite/die-pof/gi-pof-aus-fluorpolymer.html}
(zuletzt aufgerufen am 13.08.2015)}

\bibitem{poflan}{Ohland, Günther: Verkabelung mit polymeren optischen Fasern:
Datenübertragung via POF-Kabel, in: PC Magazin (05.11.2013) \newline
\url{http://www.pc-magazin.de/ratgeber/optische-polymer-faser-datenuebertragung-via-pof-kabel-1580596.html}
(zuletzt aufgerufen am 12.09.2015)}

\bibitem{pofacprinzip}{Technische Hochschule Nürnberg: POF Application Center:
Prinzip der POF \newline
\url{http://www.pofac.info/startseite/die-pof/prinzip-der-pof.html} (zuletzt
aufgerufen am 13.08.2015)}

\bibitem{pofacsi}{Technische Hochschule Nürnberg: POF Application Center: Die
PMMA-SI-POF \newline
\url{http://www.pofac.info/startseite/die-pof/standard-pof.html} (zuletzt
aufgerufen am 13.08.2015)}

\bibitem{poflee}{Lee, Jeffrey: Discrete Multitone Modulation for Short-Range
Optical Communica\-tions, Kapitel 2.1 \newline
\url{http://alexandria.tue.nl/extra2/200613098.pdf} (zuletzt aufgerufen am
18.09.2015)}

\bibitem{pofacprofile}{Technische Hochschule Nürnberg: POF Application Center:
Brechzahlprofile \newline
\url{http://www.pofac.info/startseite/die-pof/brechzahlprofile.html} (zuletzt
aufgerufen am 13.08.2015)}

\bibitem{pofagc}{Assahi Glas Co., Ltd.: CYTOP\textsuperscript{\texttrademark} -
Technical Information \newline
\url{http://www.agcce.com/cytop-technical-information/} (zuletzt aufgerufen am
18.09.2015)}

\bibitem{poffontex}{Assahi Glas Co., Ltd.:
FONTEX\textsuperscript{\texttrademark} - New product introduction, S.10 \newline
\url{www.lucina.jp/eg_fontex/pdf/Tecnhical.pdf} (zuletzt aufgerufen am
19.09.2015)}

\bibitem{pofwuppmma}{Universität Wuppertal: Chemie und ihre Didaktik: 3.1
Polymethylmethacrylat \newline
\url{www.chemiedidaktik.uni-wuppertal.de/files/lehre/examensarbeiten/gaertner/pmma1.pdf}
(zuletzt aufgerufen am 20.09.2015)}

\bibitem{pofversuch}{Universität Wuppertal: Chemie und ihre Didaktik: 5.1.1
Herstellung von PMMA im Reagenzglas \newline
\url{www.chemiedidaktik.uni-wuppertal.de/files/lehre/examensarbeiten/gaertner/exp/pmma9.pdf}
(zuletzt aufgerufen am 02.11.2015)}

\bibitem{pofspinn}{Beckers, Markus / Schlüter, Tobias / Vad, Thomas / Gries,
Thomas / Bunge, Christian-Alexander: An overview on fabrication methods for
polymer optical fibers in: Polymer International,  \newline
\url{www.researchgate.net/profile/C_A_Bunge/publication/265646639_An_overview_on_fabrication_methods_for_polymer_optical_fibers/links/54d106950cf28959aa7a5823.pdf?inViewer=true&pdfJsDownload=true&disableCoverPage=true&origin=publication_detail}
(zuletzt aufgerufen am 28.09.2015)}

\end{thebibliography}
