\subsection{Geschichte der Compact Disc}
\label{subsec:cdgeschichte}

Die Entwicklung von optischen Datenträgern für laserbasierte Auslesesysteme
begann in den 70er Jahren des \nolbreaks{20. Jahrhunderts}. 1975 brachte Philips
den ersten optischen Datenträger auf den Markt, die Laservision Videodisc (siehe
\autoref{fig:videodisc}). Sie sollte eine Alternative zum VHS-Videosystem (siehe
\autoref{fig:vhs}) darstellen, war jedoch aufgrund der geringen Verkaufszahlen
nicht erfolgreich.

\ifthenelse{\boolean{showPics}}{
    \begin{figure}[h]
        \begin{center}
            \begin{minipage}[t]{0.4\textwidth}
                \begin{center}
                    \includegraphics[height=0.1\textheight]{Bilder/Optische_Datentraeger_Die_Compact_Disc/Geschichte/videodisc.png}
                    \caption[Laservision Videodisc \newline \url{http://www.sciencemuseum.org.uk/online_science/explore_our_collections/objects/index/smxg-8095649} (zuletzt aufgerufen am 19.09.2015)]{Laservision Videodisc}
                    \label{fig:videodisc}
                \end{center}
            \end{minipage}
            \hspace{0.025\textwidth}
            \begin{minipage}[t]{0.4\textwidth}
                \begin{center}
                    \includegraphics[height=0.1\textheight]{Bilder/Optische_Datentraeger_Die_Compact_Disc/Geschichte/vhs.png}
                    \caption[VHS cassette \newline \url{https://upload.wikimedia.org/wikipedia/commons/6/67/VHS-cassette.jpg} (zuletzt aufgerufen am 19.09.2015)]{VHS Cassette}
                    \label{fig:vhs}
                \end{center}
            \end{minipage}
        \end{center}
    \end{figure}
}{}

Auf Basis der Videodisc entwickelte Philips bis 1977 die Compact Disc Digital
Audio mit einem Durchmesser von \nolbreaks{11,5 cm} und einer Spielzeit von 60
Minuten. Ab 1979 arbeiteten Philips und Sony dann an einem gemeinsamen, noch
heute gültigen CD-Standard mit einem Durchmesser von \nolbreaks{12 cm} und einer
75-minütigen Spielzeit. \cite{cds}

Zunächst scheiterten jedoch die Verhandlungen über Musikrechte für CDs, da die
Schallplattenproduzenten rückläufige Verkaufszahlen befürchteten. Deshalb
konzentrierten sich Philips und Sony auf klassische Musik, da die beiden
Unternehmen zudem annahmen, dass Klassikliebhaber bereit wären, mehr Geld für
eine bessere Klangqualität auszugeben.

Der CD gelang der Durchbruch während der Salzburger Festspiele im Jahr 1981.
Philips und Sony konnten ein Publikum aus Musikkritikern von dem
Zukunftspotential der CD überzeugen. Die Begeisterung der Kritiker erfasste die
internationale Musikszene und bis März 1982 hatten acht Schallplattenfirmen
Verträge mit Philips und Sony unterschrieben. Die Markteinführung erfolgte im
August 1982. \cite{cuz}

Innerhalb weniger Jahre verdrängte die CD die Schallplatte fast komplett vom
Markt. \autoref{fig:umsatzcd} zeigt die Umsatzentwicklung der deutschen
Musikindustrie mit den jeweiligen Tonträgern. Der Umsatz mit Vinylschallplatten,
1980 mit ca. 760 Mio. \euro{} noch das umsatzstärkste Medium, sank bis 1987 erst
all­mäh­lich und dann bis 1992 rapide. Im selben Zeitraum stieg der Umsatz mit
CDs massiv an und erreichte im Jahr 1997 ein Maximum von 2,3 Mrd. \euro{}. Ab
1997 war der Umsatz mit CDs rückläufig, da Privatpersonen selbst CDs \glqq
brennen\grqq{} konnten und zunehmend legale und illegale Angebote im Internet
nutzten.

\ifthenelse{\boolean{showPics}}{
    \begin{figure}[H]
        \begin{center}
            \begin{minipage}[t]{\textwidth}
                \begin{center}
                    \includegraphics[width=\textwidth]{Bilder/Optische_Datentraeger_Die_Compact_Disc/Geschichte/cdumsatz.png}
                    \caption[Umsatzentwicklung der deutschen Musikindustrie von 1984 bis 2013 \newline verändert nach \url{http://www.musikindustrie.de/uploads/media/140325\_BVMI\_2013\_Jahrbuch\_ePaper\_V02.pdf} S.7 (zuletzt aufgerufen am 03.08.2015)]{Umsatzentwicklung der deutschen Musikindustrie von 1984 bis 2013}
                    \label{fig:umsatzcd}
                \end{center}
            \end{minipage}
        \end{center}
    \end{figure}
}{}
