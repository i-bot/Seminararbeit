\subsection{Funktionsweise von optischen Datenträgern}
\label{subsec:cdfunktionsweise}

Die Compact Disc besteht aus einer Polycarbonatscheibe und einer 40-80 nm
dicken, zumeist aus Aluminium bestehenden Reflexionsschicht. Zusätzlich gibt es
noch eine 10-20 µm dicke Lackschicht, welche die Reflexionsschicht schützt.
Zusammen genommen ergibt dies eine Höhe von ca. 1,2 mm (siehe
\autoref{fig:cdquer}). Außerdem lässt sich eine Einbuchtung (\textit{pit})
erkennen, die sich von der Grundebene (\textit{land}) abhebt. \cite{cfcd}

\ifthenelse{\boolean{showPics}}{
    \begin{figure}[h]
        \begin{center}
            \begin{minipage}[t]{\textwidth}
                \begin{center}
                    \includegraphics[height=0.1\textheight]{Bilder/Optische_Datentraeger_Die_Compact_Disc/Funktionsweise/cdquer.png}
                    \caption[Querschnitt einer CD \newline \url{http://daten.didaktikchemie.uni-bayreuth.de/umat/cd_dvd/cd-ausschnitt.gif} (zuletzt aufgerufen am 07.08.2015)]{Querschnitt einer CD}
                    \label{fig:cdquer}
                \end{center}
            \end{minipage}
        \end{center}
    \end{figure}
}{}

Die Abtasteinheit, bestehend aus einem Laser und einem Photodetektor, liest die
Informationen auf einer spiralförmigen Spur von innen nach außen aus. Dafür
wandelt der Fotodetektor das Laserlicht, das durch die Polycarbonatscheibe geht
und von der Reflexionsschicht zurückgeworfen wird, in Strom um, wobei die
Stromstärke abhängig von der Lichtintensität ist. Überstreift der Laserstrahl
ein \textit{pit}, nehmen die Lichtintensität und damit auch die Stromstärke ab.
Da der Durchmesser des Laserstrahls größer als die Pitbreite ist (siehe
\autoref{fig:cdrem}), trifft dieser nur teilweise auf den \textit{pit}. Der Rest
des Lichtes trifft etwas früher auf den umgebenden \textit{land}. Die
Verschiebung der Strahlen gegeneinander, welche man in \autoref{fig:cdlaser}
erkennen kann, resultiert in einer Intensitätsabnahme, da es aufgrund der
destruktiven Interferenz\footnote{Wellenberg und Wellental zweier Lichtstrahlen
mit gleicher Frequenz und Amplitude treffen aufeinander und heben sich
gegenseitig auf.} zu einer teilweisen Auslöschung des Laserlichts kommt.
\autoref{fig:cdstrom} zeigt einen möglichen Stromstärkenverlauf. Den Übergängen
zwischen \textit{pit} und \textit{land} wird der Binärwert 1 zugewiesen. Dies
entspricht dem Schnittpunkt zwischen der Mittelwertlinie und dem Graph der
Stromstärke in \autoref{fig:cdstrom}. \cite{cdp}

\ifthenelse{\boolean{showPics}}{
    \begin{figure}[h]
        \begin{center}
            \begin{minipage}[t]{0.3\textwidth}
                \begin{center}
                    \includegraphics[height=0.1\textheight]{Bilder/Optische_Datentraeger_Die_Compact_Disc/Funktionsweise/remcd.png}
                    \caption[Laserlicht auf einem \textit{pit} unter einem Rasterelektronenmikroskop \newline Roth, Klaus: CD, DVD \& Co.: Die Chemie der schillernden Scheiben, in: Chemie in unserer Zeit (41/2007), S. 340]{Laserlicht auf einem \textit{pit} unter einem Rasterelektronenmikroskop}
                    \label{fig:cdrem}
                \end{center}
            \end{minipage}
            \hspace{0.025\textwidth}
            \begin{minipage}[t]{0.3\textwidth}
                \begin{center}
                    \includegraphics[height=0.1\textheight]{Bilder/Optische_Datentraeger_Die_Compact_Disc/Funktionsweise/cdlaser.png}
                    \caption[destruktive Interferenz von Laserlicht bei einem \textit{pit} \newline \url{http://www.muenster.de/~asshoff/physik/cd/image50.gif} (zuletzt aufgerufen am 07.08.2015)]{destruktive Interferenz von Laserlicht bei einem \textit{pit}}
                    \label{fig:cdlaser}
                \end{center}
            \end{minipage}
            \hspace{0.025\textwidth}
            \begin{minipage}[t]{0.3\textwidth}
                \begin{center}
                    \includegraphics[width=0.9\textwidth]{Bilder/Optische_Datentraeger_Die_Compact_Disc/Funktionsweise/cdstrom.png}
                    \caption[Stromstärkenverlauf \newline \url{http://www.muenster.de/~asshoff/physik/cd/image51.gif} (zuletzt aufgerufen am 07.08.2015)]{Stromstärkenver-lauf}
                    \label{fig:cdstrom}
                \end{center}
            \end{minipage}
        \end{center}
    \end{figure}
}{}

Neben der CD sind gebräuchliche optische Datenträger die DVD (Digital Video
Disc) und die Blue-ray Disc, welche für die Speicherung von Filmen verwendet
werden. Beide zeichnen sich durch eine höhere Speicherkapazität aufgrund von
geringeren Spurabständen und mehrlagigen Informationsebenen aus.
