\subsection{Funktionsweise}
\label{subsec:cdfunktionsweise}

Die Compact Disc besteht aus einer Polycarbonatscheibe und einer 40-80nm dicken
zumeist aus Aluminium bestehenden Reflexionsschicht. Zusätzlich gibt es noch
eine 10-20µm dicke Lackschicht welche die Reflexionsschicht schützt.
Zusammengenommen ergibt dies eine Höhe von ca. 1,2mm wie man in ABB.QUERSCHNITT
erkennen kann. Außerdem lässt sich eine Ausbuchtung(\textit{pit}), welche sich
von der Grundebene(\textit{land}) abhebt, erkennen. \cite{cfcd}

Die Abtasteinheit, welche aus einem Laser und einem Fotodetektor besteht, liest
spiralförmig von innen nach außen die Informationen auf der CD aus. Dafür
wandelt der Fotodetektor das von der Reflexionsschicht zurückgeworfene
Laserlichtes in Strom um, wobei die Stromstärke abhängig von der Lichtintensität
ist. Eine Abnahme der Lichtintensität kann man beim Überstreifen des
Laserstrahls von einem \textit{pit} erkennen und resultiert in einer Abnahme der
Stromstärke. Da der Durchmesser des Laserstrahls größer ist als die Pitbreite
(siehe ABB.REMLASERCD) trifft dieser nur teilweise auf den \textit{pit}. Der
Rest des Lichtes trifft mit geringer Verzögerung auf den Umgebenden
\textit{land}. Die Verschiebung der Strahlen gegeneinander, welche man in
ABB.SKIZZEPITLAND erkennen kann, resultiert in einer Intensitätsabnahme bedingt
durch die teilweisige Auslöschung des Laserlichtes aufgrund der destruktiven
Interferenz\footnote{Wellenberg und Wellental zweier Lichtstrahlen mit gleicher
Frequenz und Amplitude treffen aufeinander.}. ABB.GRAPHSTROM zeigt einen
möglichen Stromstärkenverlauf. \cite{cdp}
