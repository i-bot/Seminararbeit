\subsubsection{Versuch: Nachweis der hohen Lichtdurchlässigkeit von PC und der einfachen Weiterverarbeitung}

Material:
\begin{itemize*}
    \item handelsübliche CD, Schere, 2 Glasschalen, Plätzchenform, Pinzette, Alufolie, Heizplatte
    \item Chemikalien: Salpetersäure (Sicherheitshinweis: CAS Nr. 7697-37-2 (\autoref{fig:cdbrandfoerdernd}, \autoref{fig:cdaetzwirkung}))
    \item Chemikalien: \\
    \begin{xtabular}{ll}
    Salpetersäure: & Sicherheitshinweis: CAS Nr. 7697-37-2 \\
    & (brandfördernder und ätzender Stoff) \\
    \end{xtabular}
\end{itemize*}

Schutzvorkehrungen:
\begin{itemize}
    \item Abzug, Schutzkleidung, Handschuhe, Schutzbrille
\end{itemize}

Versuchsablauf:
\begin{enumerate*}
    \item Die CD wird in eine der Glasschalen gelegt und mit Salpetersäure übergossen (siehe \autoref{fig:cdsalpeter}). Dies muss unter einem Abzug geschehen, da nitrose Gase entstehen.
    \item Nach kurzer Zeit \glqq quellen\grqq{} die Lack- und die Aluminiumschicht auf (siehe \autoref{fig:cdquillt}) und lassen sich mithilfe der Pinzette entfernen.
    \item Die \glqq gehäutete\grqq{} CD wird nun in die zweite Glasschale gelegt und vorsichtig unter dem Wasserhahn abgespült. \autoref{fig:cdblank} zeigt die resultierende Polycarbonatscheibe.
    \item Die Lichtdurchlässigkeit der Polycarbonatscheibe wird überprüft, indem die Scheibe in ein CD-Laufwerk eingelegt wird.
    \item Die Polycarbonatscheibe wird nun in kleine Stücke zerschnitten, die nicht größer als 1 cm² sind (siehe \autoref{fig:cdzerschnitten}).
    \item Die Polycarbonatstücke werden ca. 1 cm hoch in die Plätzchenform gefüllt, welche sich auf der mit Aluminiumfolie bedeckten Heizplatte befindet (siehe \autoref{fig:cdschmelzen}).
    \item Die Heizplatte wird auf über 250°C erhitzt und man wartet, bis das Polycarbonat zu einer gleichmäßigen Oberfläche zerflossen ist.
    \item Die Heizplatte wird nun ausgeschaltet. Nach dem Abkühlen kann das \glqq Polycarbonatplätzchen\grqq{} aus der Form gebrochen werden (siehe \autoref{fig:cdplaetzchen}).
\end{enumerate*}

\begin{figure}[h]
    \begin{center}
        \begin{minipage}[t]{0.4\textwidth}
            \begin{center}
                \includegraphics[height=0.1\textheight]{Bilder/Optische_Datentraeger_Die_Compact_Disc/Traegermaterial_Polycarbonat/cdsalpeter.png}
                \caption[CD in Salpetersäure]{CD in Salpetersäure}
                \label{fig:cdsalpeter}
            \end{center}
        \end{minipage}
        \hspace{0.025\textwidth}
        \begin{minipage}[t]{0.4\textwidth}
            \begin{center}
                \includegraphics[height=0.1\textheight]{Bilder/Optische_Datentraeger_Die_Compact_Disc/Traegermaterial_Polycarbonat/cdquillt.png}
                \caption[\qlqq Aufgequolleney\grqq{} Lack- und Aluminiumschicht]{\glqq Aufgequollene\grqq{} Lack- und Aluminiumschicht}
                \label{fig:cdquillt}
            \end{center}
        \end{minipage}
    \end{center}
\end{figure}

\begin{figure}[h]
    \begin{center}
        \begin{minipage}[t]{0.4\textwidth}
            \begin{center}
                \includegraphics[height=0.1\textheight]{Bilder/Optische_Datentraeger_Die_Compact_Disc/Traegermaterial_Polycarbonat/cdblank.png}
                \caption[Polycarbonatscheibe]{Polycarbonatscheibe}
                \label{fig:cdblank}
            \end{center}
        \end{minipage}
        \hspace{0.025\textwidth}
        \begin{minipage}[t]{0.4\textwidth}
            \begin{center}
                \includegraphics[height=0.1\textheight]{Bilder/Optische_Datentraeger_Die_Compact_Disc/Traegermaterial_Polycarbonat/cdzerschnitten.png}
                \caption[CD in kleine Stücke zerschnitten]{CD in kleine Stücke zerschnitten}
                \label{fig:cdzerschnitten}
            \end{center}
        \end{minipage}
    \end{center}
\end{figure}

\begin{figure}[h]
    \begin{center}
        \begin{minipage}[t]{0.4\textwidth}
            \begin{center}
                \includegraphics[height=0.1\textheight]{Bilder/Optische_Datentraeger_Die_Compact_Disc/Traegermaterial_Polycarbonat/cdschmelzen.png}
                \caption[Heizplatte mit Plätzchenform]{Heizplatte mit Plätzchenform}
                \label{fig:cdschmelzen}
            \end{center}
        \end{minipage}
        \hspace{0.025\textwidth}
        \begin{minipage}[t]{0.4\textwidth}
            \begin{center}
                \includegraphics[height=0.1\textheight]{Bilder/Optische_Datentraeger_Die_Compact_Disc/Traegermaterial_Polycarbonat/cdplaetzchen.png}
                \caption[\glqq Polycarbonatplätzchen\grqq{}]{\glqq Polycarbonatplätzchen\grqq{}}
                \label{fig:cdplaetzchen}
            \end{center}
        \end{minipage}
    \end{center}
\end{figure}

Ergebnisse des Versuchs:
\begin{enumerate*}
    \item hohe Lichtdurchlässigkeit: Wird die \shorthandoff{"}"gehäutete"\shorthandon{"} Polycarbonatscheibe in das CD-Laufwerk eines Computers eingelegt, erkennt dieser nicht, dass eine CD eingelegt wurde, da der Laserstrahl ungehindert durch die Polycarbonatscheibe geht. Dies beweist ebenfalls, dass die Aluminiumschicht für die Reflektion des Laserstrahls verantwortlich ist.
    \item einfache Weiterverarbeitung: Die Polycarbonatschnipsel lassen sich ohne großen Aufwand und innerhalb kurzer Zeit in eine neue Form schmelzen. Die Qualitätsminderung liegt an Verschmutzungen und Lufteinschlüssen, die aufgrund der primitiven Methode zum Einschmelzen nicht zu vermeiden sind.
\end{enumerate*}
