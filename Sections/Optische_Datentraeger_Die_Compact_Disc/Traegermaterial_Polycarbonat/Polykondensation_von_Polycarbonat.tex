\subsubsection{Polykondensation von Polycarbonat}

Polycarbonat kann entweder über eine Umesterung oder über eine
Grenzflächenkondensation hergestellt werden \cite{cuzpe}. Beide Reaktionen
können als Sonderformen der Polykondensation eingeordnet werden, da ein
Polyester entsteht, sich aber kein Wasser abspaltet. Polycarbonat selbst ist ein
lineares Makromolekül und zählt somit zu den Thermoplasten, da die beiden Edukte
jeweils über zwei funktionale Gruppen verfügen.

\paragraph{Umesterung} Bei der Umesterung kommen die geschmolzenen Monomere
Bisphenol A und Diphenylcarbonat zum Einsatz. \autoref{rec:diphenylcarbonat}
zeigt die Herstellung von Diphenylcarbonat aus Phenol und Phosgen unter
Abspaltung von Salzsäure \cite{cuzpe}. Bisphenol A und Diphenylcarbonat
reagieren in \autoref{rec:polycarbonat} zu Polycarbonat. Zumeist wird die
Reaktion alkalisch katalysiert und findet bei Temperaturen zwischen 250 - 350°C
statt \cite{pop}. Die beiden Monomere sind zu gleichen Teilen in dem Copolymer
vertreten, da sie auf alternierende Weise angeordnet sind. Die Reaktion endet
wenn keine Monomere mehr vorhanden sind.

\ifthenelse{\boolean{showPics}}{
    \begin{figure}[H]
        \begin{center}
            \footnotesize
            \setatomsep{1.7em}

            \chemname{\chemfig{*6(-=-(-OH)=-=)}}{Phenol}
            \chemsign{+}
            \chemname{\chemfig{C(=[:90]\lewis{13,O})(-[:210,1.25pt]\lewis{357,Cl})(-[:330,1.25pt]\lewis{157,Cl})}}{Phosgen}
            \chemrel{->}
            \chemname{\chemfig{C(=[:90]\lewis{13,O})(-[:210,1.25pt]\lewis{57,O}-[:150](*6(-=-=-=)))(-[:330,1.25pt]\lewis{57,O}-[:30](*6(-=-=-=)))}}{Diphenylcarbonat}
            \chemsign{+}
            \chemname{\chemfig{HCl}}{Salzsäure}

            \caption{Herstellung von Diphenylcarbonat}
            \label{rec:diphenylcarbonat}
        \end{center}
    \end{figure}

    \begin{figure}[H]
        \begin{center}
            \footnotesize
            \setatomsep{1.7em}

            \chemname{\chemfig{C(-[4]*6(-=-(-$\scriptstyle n$\,HO)=-=))(-[2]CH_3)(-[6]CH_3)(-[0]*6(-=-(-OH)=-=))}}{Bisphenol A}
            \chemsign{+ $\scriptstyle n$}
            \chemname{\chemfig{C(=[:90]\lewis{13,O})(-[:210,1.25pt]\lewis{57,O}-[:150](*6(-=-=-=)))(-[:330,1.25pt]\lewis{57,O}-[:30](*6(-=-=-=)))}}{Diphenylcarbonat}

            \vspace{10pt}

            \chemrel{->}
            \chemname{\chemfig{C(-[4]*6(-=-(-\lewis{26,O}-[@{op,0.5}]H)=-=))(-[2]CH_3)(-[6]CH_3)(-[0]*6(-=-(-\lewis{26,O}-C(=[2]\lewis{13,O})-[@{cl,0.75}]\lewis{26,O}-(*6(-=-=-=)))=-=))}}{Polycarbonat}
            \makebraces[23pt,23pt]{n}{op}{cl}
            \chemsign{+ $\scriptstyle (2n-1)$}
            \chemname{\chemfig{*6(-=-(-OH)=-=)}}{Phenol}

            \caption{Umesterung: Polykondensation von Polycarbonat}
            \label{rec:polycarbonat}
        \end{center}
    \end{figure}
}{}

Die \autoref{rec:deprotonierung} bis \autoref{rec:abspaltung} zeigen den ersten
Umesterungsvorgang, also den Beginn des Kettenwachstums im basischen Milieu. Als
erstes findet die Deprotonierung (siehe \autoref{rec:deprotonierung}) von
Bisphenol A statt. Dabei spaltet sich ein Proton (H\textsuperscript{+}) von
einer der Hydroxygruppen ab und lagert sich an das Hydroxidion an. Durch diesen
Vorgang wird das Bisphenol A negativ geladen und das Hydroxidion wird zu Wasser
protoniert. Die Umesterung selbst beginnt mit dem nukleophilen Angriff (siehe
\autoref{rec:nukleophilerangriff}) des Bisphenol A-Anions auf das
Diphenylcarbonat. Dabei lagert sich das negativ geladene Sauerstoffatom an das
positivpolarisierte Kohlenstoffatom des Diphenylcarbonats an. Aufgrund der
Vierbindigkeit des C-Atoms \glqq klappt\grqq{} eine der Elektronenpaarbindungen
des doppeltgebunden Sauerstoffatoms zu dem O-Atom. Durch weiteres \glqq
Herumklappen\grqq{} von Elektronenpaarbindungen, wie in \autoref{rec:abspaltung}
dargestellt, kann eine der Phenylgruppen in Form eines Phenolations abgespalten
werden. Dabei bildet sich die Doppelbindung zwischen dem Sauerstoff- und dem
Kohlenstoffatom zurück und die Bindung zwischen einer der beiden Phenylgruppen
und dem C-Atom \glqq klappt\grqq{} zum O-Atom. Das entstandene Dimer kann
entweder einen eigenen nukleophilen Angriff auf ein Diphenylcarbonat oder auf
ein ebenfalls durch Umesterung entstandenes Oligomer bzw. Polymer durchführen,
vorausgesetzt die verbliebene Hydroxygruppe wurde zuvor deprotoniert, oder es
wird selbst durch ein Bisphenonal A-Anion oder durch ein deprotoniertes Oligomer
bzw. Polymer angegriffen. In beiden Fällen kommt es zu einer Umesterung und die
Kette wächst durch die Wiederholung dieser Schritte zu einem Polycarbonat an.

\ifthenelse{\boolean{showPics}}{
    \begin{figure}[H]
        \begin{center}
            \footnotesize
            \setatomsep{1.7em}

            \chemname{\chemfig{C(-[4]*6(-=-(-HO)=-=))(-[2]CH_3)(-[6]CH_3)(-[0]*6(-=-(-O@{pr1}H)=-=))}}{Bisphenol A}
            \chemsign{+}
            \chemname{\chemfig{@{ak1}OH(-[:135,.5,,,draw=none]\fsscrm)}}{Hydroxidion}
            \chemmove[->,shorten <=2pt]{\draw[shorten >=2pt](pr1).. controls +(90:1cm) and +(90:1cm).. (ak1);}
            \chemrel{->}
            \chemname{\chemfig{C(-[4]*6(-=-(-HO)=-=))(-[2]CH_3)(-[6]CH_3)(-[0]*6(-=-(-\lewis{026,O}(-[:45,.7,,,draw=none]\fsscrm))=-=))}}{Bisphenol A-Anion}
            \chemsign{+}
            \chemname{\chemfig{H_2O}}{Wasser}

            \caption{Umesterung: Deprotonierung von Bisphenol A durch Hydroxidionen}
            \label{rec:deprotonierung}
        \end{center}
    \end{figure}

    \begin{figure}[H]
        \begin{center}
            \footnotesize
            \setatomsep{1.7em}

            \chemname{\chemfig{C(-[4]*6(-=-(-HO)=-=))(-[2]CH_3)(-[6]CH_3)(-[0]*6(-=-(-@{np2}\lewis{026,O}(-[:45,.7,,,draw=none]\fsscrm))=-=))}}{Bisphenol A-Anion}
            \chemsign{+}
            \chemname{\chemfig{@{ep2}C(-[:25,.7,,,draw=none]\fdelp)(=[@{db2}:90]@{o2}\lewis{13,O})(-[:210]\lewis{57,O}-[:150](*6(-=-=-=)))(-[:330]\lewis{57,O}-[:30](*6(-=-=-=)))}}{Diphenylcarbonat}
            \chemmove[->,shorten <=2.5pt]{
                \draw[shorten >=2.5pt](np2).. controls +(90:1cm) and +(135:2.5cm).. (ep2);
                \draw[shorten >=6pt](db2).. controls +(0:1cm) and +(90:1cm).. (o2);}

            \vspace{10pt}

            \chemrel{->}
            \chemname{\chemfig{C(-[4]*6(-=-(-HO)=-=))(-[2]CH_3)(-[6]CH_3)(-[0]*6(-=-(-\lewis{26,O}-C(-[2]\lewis{024,O}(-[:45,.7,,,draw=none]\fsscrm))(-[0]\lewis{26,O}-(*6(=-=-=-)))(-[6]\lewis{04,O}-[6](*6(=-=-=-))))=-=))}}{}

            \caption{Umesterung: nukleophiler Angriff}
            \label{rec:nukleophilerangriff}
        \end{center}
    \end{figure}

    \begin{figure}[H]
        \begin{center}
            \footnotesize
            \setatomsep{1.7em}

            \chemname{\chemfig{C(-[4]*6(-=-(-HO)=-=))(-[2]CH_3)(-[6]CH_3)(-[0]*6(-=-(-\lewis{26,O}-C(-[@{ndb3}2]@{o3}\lewis{024,O}(-[:45,.7,,,draw=none]\fsscrm))(-[0]\lewis{26,O}-(*6(=-=-=-)))(-[@{b3}6]@{p3}\lewis{04,O}-[6](*6(=-=-=-))))=-=))}}{}
            \chemmove[->,shorten <=2.5pt]{
                \draw[shorten >=2.5pt](b3).. controls +(0:1cm) and +(315:1cm).. (p3);
                \draw[shorten >=2.5pt](o3).. controls +(0:0.5cm) and +(0:0.5cm).. (ndb3);}

            \vspace{10pt}

            \chemrel{->}
            \chemnameinit{}
            \chemname{\chemfig{C(-[4]*6(-=-(-HO)=-=))(-[2]CH_3)(-[6]CH_3)(-[0]*6(-=-(-\lewis{26,O}-C(=[2]\lewis{13,O})-\lewis{26,O}-(*6(-=-=-=)))=-=))}}{Dimer}
            \chemsign{+}
            \chemname{\chemfig{*6(-=-(-\lewis{137,O}(-[:0,.7,,,draw=none]\fsscrm))=-=)}}{Phenolation}

            \caption{Umesterung: Abspaltung eines Phenolation}
            \label{rec:abspaltung}
        \end{center}
    \end{figure}
}{}

\paragraph{Grenzflächenkondensation} Die zweite Möglichkeit, Polycarbonat
herzustellen, besteht in der Grenzflächenkondensation. Hierfür wird Bisphenol A
in eine wässrige basische Lösung und Phosgen in eine organische Lösung versetzt
\cite{cuzpe}. Als Base für die wässrige Lösung kann z.B. Natriumhydroxid dienen.
Durch die Deprotonierung von Bisphonol A durch Natriumhydroxid (siehe
\autoref{rec:deprotonierung2}) enthält die wässrige Lösung Bisphenol A-Anionen
und Natriumionen. Durch die negative Ladung des Bisphenol A erhöht sich dessen
Reaktivität und begünstigt das Kettenwachstum. \autoref{rec:polycarbonat2} zeigt
die Polymerbildung durch eine Polykondensation zwischen den Bisphenol A-Anionen
und Phosgen an der Grenzfläche zwischen der basischen und organischen Phase.
Dabei spalten sich Chloridionen ab, welche mit den Natriumionen als
Natriumchlorid ausfallen. Die Reaktion in \autoref{rec:polycarbonat2} findet in
Form eines nukleophilen Angriffes statt, der dem zwischen dem Bispenol A-Anion
und Diphenylcarbonat in \autoref{rec:nukleophilerangriff} und
\autoref{rec:abspaltung} ähnelt (das Phosgen übernimmt die Funktion des
Diphenylcarbonats und es spaltet sich statt eines Phenolations ein Chloridion
ab). Das Polycarbonat lässt sich durch Verdampfen des organischen
Lösungsmittels, in welchem sich das Polymer ansammelt, gewinnen \cite{garoo}.

\ifthenelse{\boolean{showPics}}{
    \begin{figure}[H]
        \begin{center}
            \footnotesize
            \setatomsep{1.7em}

            \chemname{\chemfig{C(-[4]*6(-=-(-HO)=-=))(-[2]CH_3)(-[6]CH_3)(-[0]*6(-=-(-OH)=-=))}}{Bisphenol A}
            \chemsign{+}
            \chemname{\chemfig{2Na(-[:45,.5,,,draw=none]\fsscrp)(-[:0,1.2,,,draw=none])OH(-[:135,.5,,,draw=none]\fsscrm)}}{Natriumhydroxid}

            \vspace{10pt}

            \chemrel{->}
            \chemname{\chemfig{C(-[4]*6(-=-(-\lewis{246,O}(-[:135,.7,,,draw=none]\fsscrm)(-[:180,1.4,,,draw=none])Na(-[:45,.6,,,draw=none]\fsscrp))=-=))(-[2]CH_3)(-[6]CH_3)(-[0]*6(-=-(-\lewis{026,O}(-[:45,.7,,,draw=none]\fsscrm)(-[:0,1.4,,,draw=none])Na(-[:135,.6,,,draw=none]\fsscrp))=-=))}}{Bisphenol A-Anion, Natriumionen}
            \chemsign{+}
            \chemname{\chemfig{2H_2O}}{Wasser}

            \caption{Grenzflächenkondensation: Deprotonierung von Bisphenol A durch Natriumhydroxid}
            \label{rec:deprotonierung2}
        \end{center}
    \end{figure}

    \begin{figure}[H]
        \begin{center}
            \footnotesize
            \setatomsep{1.7em}

            \chemname{\chemfig{C(-[4]*6(-=-(-\lewis{246,O}(-[:135,.7,,,draw=none]\fsscrm)(-[:180,1.4,,,draw=none])$\scriptstyle n$\,Na(-[:45,.6,,,draw=none]\fsscrp))=-=))(-[2]CH_3)(-[6]CH_3)(-[0]*6(-=-(-\lewis{026,O}(-[:45,.7,,,draw=none]\fsscrm)(-[:0,1.4,,,draw=none])Na(-[:135,.6,,,draw=none]\fsscrp))=-=))}}{Bisphenol A-Anion, Natriumionen}
            \chemsign{+ $\scriptstyle n$}
            \chemname{\chemfig{C(=[:90]\lewis{13,O})(-[:210,1.25pt]\lewis{357,Cl})(-[:330,1.25pt]\lewis{157,Cl})}}{Phosgen}

            \vspace{10pt}

            \chemrel{->}
            \chemname{\chemfig{C(-[4]*6(-=-(-\lewis{26,O}-[@{op,0.75}])=-=))(-[2]CH_3)(-[6]CH_3)(-[0]*6(-=-(-\lewis{26,O}-C(=[2]\lewis{13,O})-[@{cl,0.75}])=-=))}}{Polycarbonat}
            \makebraces[23pt,23pt]{n}{op}{cl}
            \chemsign{+ $\scriptstyle 2n$}
            \chemname{\chemfig{Na(-[:45,.6,,,draw=none]\fsscrp)(-[:0,1.3,,,draw=none])Cl(-[:135,.6,,,draw=none]\fsscrm)}}{Natriumchlorid}

            \caption{Grenzflächenkondensation: Polykondensation von Polycarbonat}
            \label{rec:polycarbonat2}
        \end{center}
    \end{figure}
}{}
