\subsubsection{Vorteile von Polycarbonat gegenüber Polymethylmethacrylat}

Polycarbonat wurde 1953 von dem bei der Firma Bayer angestellten Chemiker
Hermann Schnell entdeckt. Der neue Kunststoff wurde später unter dem Namen
Makrolon\textsuperscript{\textregistered} vermarktet. \cite{cuzpc}

Nachdem Philips seine ersten CD-Prototypen hergestellt hatte, suchte man nach
einem Trägermaterial, welches für die Massenproduktion mittels
Spritzgussverfahren geeignet ist. Hierfür ist Polycarbonat nahezu perfekt. Die
niedrige Viskosität ermöglicht eine fehlerfreie Übertragung des
\textit{pit}-Musters von der Metallmatrize auf das flüssige Polycarbonat. Hohe
Transparenz und ein konstanter Brechungsindex\footnote{Verhältnis der
Lichtgeschwindigkeit zur Ausbreitungsgeschwindigkeit von Licht im untersuchten
Material} erlauben ein unabgeschwächtes Durchdringen des Laserstrahls durch das
Trägermaterial. Eine hohe Erweichungstemperatur bei ca. 149°C \cite{cuzpc2} und
Resistenz gegenüber physikalischen Belastungen machen Polycarbonat
alltagstauglich. \cite{cfcd}

\autoref{fig:cdpcpmma} vergleicht Polycarbonat (PC) und Polymethylmethacrylat
(PMMA) in Bezug auf Eigenschaften, die für die Herstellung und Nutzung der CD
relevant sind. Die Eignung nimmt in den jeweiligen Punkten von innen nach außen
zu. PMMA schneidet in fast allen Punkten mit Bestnoten ab, ist jedoch PC in den
Kategorien Wärmeformbeständigkeit und Wasseraufnahme weit unterlegen. Dies sowie
das gute Abschneiden bei den anderen Eigenschaften machen Polycarbonat zum
bevorzugten Kunststoff für die CD-Produktion.

\ifthenelse{\boolean{showPics}}{
    \begin{figure}[h]
        \begin{center}
            \begin{minipage}[t]{\textwidth}
                \begin{center}
                    \includegraphics[height=0.15\textheight]{Bilder/Optische_Datentraeger_Die_Compact_Disc/Traegermaterial_Polycarbonat/cdpcpmma.png}
                    \caption[Vergleich zwischen PC und PMMA \newline Roth, Klaus: CD, DVD \& Co.: Die Chemie der schillernden Scheiben, in: Chemie in unserer Zeit (41/2007), S. 337]{Vergleich zwischen PC und PMMA}
                    \label{fig:cdpcpmma}
                \end{center}
            \end{minipage}
        \end{center}
    \end{figure}
}{}
