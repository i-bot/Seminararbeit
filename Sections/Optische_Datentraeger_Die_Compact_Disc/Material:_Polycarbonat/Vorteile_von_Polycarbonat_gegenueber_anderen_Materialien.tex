\subsubsection{Vorteile von Polycarbonat gegenüber anderen Materialien}

Polycarbonat wurde 1953 von dem, bei der Firma Bayer angestellten, Chemiker
Hermann Schnell entdeckt. Der neue Kunststoff wurde später unter dem Namen
Makrolon\textsuperscript{\textregistered} vermarktet.

Nachdem seine ersten CD-Prototypen hergestellt hat, suchte man nach
Trägermaterial, welches für die Massenproduktion mittels des
Spritzgussverfahrens geeignet ist. Hierfür ist Polycarbonat nahezu perfekt. Die
niedriege Viskosität ermöglicht eine fehlerfreie übertragung der Pitstruktur von
der Matrize auf die Polycarbonatscheibe. Hohe Transparenz und ein konstater
Brechungsindex\footnote{Verhältnis der Lichtgeschwindigkeit und der
Ausbreitungsgeschwindigkeit von Licht im untersuchten Material} erlauben ein
unabgeschwächtes Durchdringen des Laserstrahls durch das Trägermaterial. Eine
hohe Erweichungstemperatur\footnote{ca. 149°C} und Resistenz gegenüber
physikalischen Belastungen machen Polycarbonat ebenfalls alltagstauglich.

Die \autoref{fig:cdpcpmma} vergleicht Polycarbonat(PC) und
Polymethylmethacrylat(PMMA) im Bezug auf Eigenschaften, die für die Herstellung
und Benutzung der CD von Vorteil sind. Die Eignung nimmt in den jeweiligen
Punkten von innen nach außen zu. PMMA schneidet in fast allen Punkten mit
Bestnote ab. Jedoch machen ein insgesamt mittelmäßiges Abschneiden von PC und
bessere Eigenschaften in den Kategorien Wärmeformbeständigkeit und geringe
Wasseraufnahme als Polymethylmethacrylat Polycarbonat zum bevorzugten Kunststoff
für die CD-Produktion.

\begin{figure}[h]
  \begin{center}
      \begin{minipage}[t]{\textwidth}
        \begin{center}
            \includegraphics[height=0.1\textheight]{Bilder/Optische_Datentraeger_Die_Compact_Disc/Material:_Polycarbonat/cdpcpmma.png}
            \caption[Vergleich zwischen PC und PMMA \newline Roth, Klaus: CD, DVD \& Co.: Die Chemie der schillernden Scheiben, in: Chemie in unserer Zeit (41/2007), S. 337]{Vergleich zwischen PC und PMMA}
            \label{fig:cdpcpmma}
        \end{center}
      \end{minipage}
  \end{center}
\end{figure}
