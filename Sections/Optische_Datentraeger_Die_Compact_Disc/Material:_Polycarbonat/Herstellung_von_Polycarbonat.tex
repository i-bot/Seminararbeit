\subsubsection{Herstellung von Polycarbonat}

Polycarbonat kann entweder über eine Umesterung oder über eine
Grenzflächenkondensation hergestellt werden. Beide Reaktionen können als
Sonderformen der Polykondensation eingeordnet werden, da ein Polyester entsteht,
sich aber kein Wasser abspaltet. Polycarbonat selbst ist ein lineares
Makromolekül und zählt somit zu den Thermoplasten.

Bei der Umesterung kommen die, in Schmelze versetzten, Monomere Bisphenol A und
Diphenylcarbonat zum Einsatz. \autoref{rec:diphenylcarbonat} zeigt die
Herstellung von Diphenylcarbonat aus Phenol und Phosgen unter Abspaltung von
Salzsäure. Bisphenol A und Diphenylcarbonat reagieren in
\autoref{rec:polycarbonat} Milieu zu Polycarbonat. Die beiden Monomere sind zu
gleichen Teilen in dem Copolymere vertreten, da sie auf alternierende Weise
angeordnet sind. Die Reaktion endet wenn keine Monomere mehr vorhanden sind.

\begin{figure}[h]
    \begin{center}
        \footnotesize
        \setatomsep{1.7em}

        \chemname{\chemfig{*6(-=-(-OH)=-=)}}{Phenol}
        \chemsign{+}
        \chemname{\chemfig{C(=[:90]\lewis{13,O})(-[:210]\lewis{357,Cl})(-[:330]\lewis{157,Cl})}}{Phosgen}
        \chemrel{->}
        \chemname{\chemfig{C(=[:90]\lewis{13,O})(-[:210]\lewis{57,O}-[:150](*6(-=-=-=)))(-[:330]\lewis{57,O}-[:30](*6(-=-=-=)))}}{Diphenylcarbonat}
        \chemsign{+}
        \chemname{\chemfig{HCl}}{Salzsäure}

        \caption{Reaktion: Diphenylcarbonat}
        \label{rec:diphenylcarbonat}
    \end{center}
\end{figure}

\begin{figure}[h]
    \begin{center}
        \footnotesize
        \setatomsep{1.7em}

        \chemname{\chemfig{C(-[4]*6(-=-(-$\scriptstyle n$\,HO)=-=))(-[2]CH_3)(-[6]CH_3)(-[0]*6(-=-(-OH)=-=))}}{Bisphenol A}
        \chemsign{+ $\scriptstyle n$}
        \chemname{\chemfig{C(=[:90]\lewis{13,O})(-[:210]\lewis{57,O}-[:150](*6(-=-=-=)))(-[:330]\lewis{57,O}-[:30](*6(-=-=-=)))}}{Diphenylcarbonat}

        \vspace{10pt}

        \chemrel{->}
        \chemname{\chemfig{C(-[4]*6(-=-(-\lewis{26,O}-[@{op,0.75}]H)=-=))(-[2]CH_3)(-[6]CH_3)(-[0]*6(-=-(-\lewis{26,O}-C(=[2]\lewis{13,O})-[@{cl,0.75}]\lewis{26,O}-(*6(-=-=-=)))=-=))}}{Polycarbonat}
        \makebraces[23pt,23pt]{n}{op}{cl}
        \chemsign{+ $\scriptstyle (2n-1)$}
        \chemname{\chemfig{*6(-=-(-OH)=-=)}}{Phenol}

        \caption{Reaktion: Polycarbonat}
        \label{rec:polycarbonat}
    \end{center}
\end{figure}

Die Deprotonierung \autoref{rec:deprotonierung} findet im basischen Milieu statt. Dabei spaltet sich ein Proton (H\textsuperscript{+}) vom einer der Hydroxygruppen ab und lagert sich an das Hydroxidion an. Durch diesen Vorgang wird das Bisphenol A negativ geladen und das Hydroxidion wird zu Wasser protoniert. Die Umesterung beginnt mit dem nukleophilen Angriff \autoref{rec:nukleophilerangriff} des Bisphenol A-Anions auf das Diphenylcarbonat.

\begin{figure}[h]
    \begin{center}
        \footnotesize
        \setatomsep{1.7em}

        \chemname{\chemfig{C(-[4]*6(-=-(-HO)=-=))(-[2]CH_3)(-[6]CH_3)(-[0]*6(-=-(-O@{pr1}H)=-=))}}{Bisphenol A}
        \chemsign{+}
        \chemname{\chemfig{@{ak1}OH(-[:135,.5,,,draw=none]\fsscrm)}}{Hydroxidion}
        \chemmove[->,shorten <=2pt]{\draw[shorten >=2pt](pr1).. controls +(90:1cm) and +(90:1cm).. (ak1);}
        \chemrel{->}
        \chemname{\chemfig{C(-[4]*6(-=-(-HO)=-=))(-[2]CH_3)(-[6]CH_3)(-[0]*6(-=-(-\lewis{026,O}(-[:45,.7,,,draw=none]\fsscrm))=-=))}}{Bisphenol A-Anion}
        \chemsign{+}
        \chemname{\chemfig{H_2O}}{Wasser}

        \caption{Reaktion: Deprotonierung}
        \label{rec:deprotonierung}
    \end{center}
\end{figure}

\begin{figure}[h]
    \begin{center}
        \footnotesize
        \setatomsep{1.7em}

        \chemname{\chemfig{C(-[4]*6(-=-(-HO)=-=))(-[2]CH_3)(-[6]CH_3)(-[0]*6(-=-(-@{np2}\lewis{026,O}(-[:45,.7,,,draw=none]\fsscrm))=-=))}}{Bisphenol A-Anion}
        \chemsign{+}
        \chemname{\chemfig{@{ep2}C(-[:25,.7,,,draw=none]\fdelp)(=[@{db2}:90]@{o2}\lewis{13,O})(-[:210]\lewis{57,O}-[:150](*6(-=-=-=)))(-[:330]\lewis{57,O}-[:30](*6(-=-=-=)))}}{Diphenylcarbonat}
        \chemmove[->,shorten <=2.5pt]{
            \draw[shorten >=2.5pt](np2).. controls +(90:1cm) and +(135:2.5cm).. (ep2);
            \draw[shorten >=6pt](db2).. controls +(0:1cm) and +(90:1cm).. (o2);}

        \vspace{10pt}

        \chemrel{->}
        \chemname{\chemfig{C(-[4]*6(-=-(-HO)=-=))(-[2]CH_3)(-[6]CH_3)(-[0]*6(-=-(-\lewis{26,O}-C(-[2]\lewis{024,O}(-[:45,.7,,,draw=none]\fsscrm))(-[0]\lewis{26,O}-(*6(=-=-=-)))(-[6]\lewis{04,O}-[6](*6(=-=-=-))))=-=))}}{}

        \caption{Reaktion: Nukleophiler Angriff}
        \label{rec:nukleophilerangriff}
    \end{center}
\end{figure}

\begin{figure}[h]
    \begin{center}
        \footnotesize
        \setatomsep{1.7em}

        \chemname{\chemfig{C(-[4]*6(-=-(-HO)=-=))(-[2]CH_3)(-[6]CH_3)(-[0]*6(-=-(-\lewis{26,O}-C(-[@{ndb3}2]@{o3}\lewis{024,O}(-[:45,.7,,,draw=none]\fsscrm))(-[0]\lewis{26,O}-(*6(=-=-=-)))(-[@{b3}6]@{p3}\lewis{04,O}-[6](*6(=-=-=-))))=-=))}}{}
        \chemmove[->,shorten <=2.5pt]{
            \draw[shorten >=2.5pt](b3).. controls +(0:1cm) and +(315:1cm).. (p3);
            \draw[shorten >=2.5pt](o3).. controls +(0:0.5cm) and +(0:0.5cm).. (ndb3);}

        \vspace{10pt}

        \chemrel{->}
        \chemnameinit{}
        \chemname{\chemfig{C(-[4]*6(-=-(-OH)=-=))(-[2]CH_3)(-[6]CH_3)(-[0]*6(-=-(-\lewis{26,O}-C(=[2]\lewis{13,O})-\lewis{26,O}-(*6(-=-=-=)))=-=))}}{Carbonat}
        \chemsign{+}
        \chemname{\chemfig{*6(-=-(-\lewis{137,O}(-[:0,.7,,,draw=none]\fsscrm))=-=)}}{Phenol}

        \caption{Reaktion: Abspaltung eines Phenolmoleküls}
        \label{rec:abspaltung}
    \end{center}
\end{figure}
