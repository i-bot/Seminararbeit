\subsection{Brechzahlprofile}
\label{subsec:pofbrechzahlprofile}

Um die Bandbreite von einem POF-Kabel zu erhöhen kommen Faser zum Einsatz deren
Brechzahlen modifiziert wurden.

\subsubsection{Stufenindexprofil (SI-POF)}

Der Stufenindex ist durchsatzschwächste Brechzahlprofil (ca. 100 Megabit/s auf
100 m, spezielle Übertragungsverfahren erlauben Bitraten im Bereich von 1
Gigabit/s). Als Kernmaterial wird Polymethylmethacrylat mit einem Durchmesser
von ca. 1 mm, welcher von einem ca. 10 µm Mantel umgeben ist, eingesetzt
\cite{pofacsi}. Altarnativ kommen auch Polystyrol und Polycarbonat zum Einsatz,
jedoch sind sie für die Datenübertragung eher ungegeignet aufgrund der
geringeren Qualität als Polymethylmethacrylat\autoref{fig:pofsi} fasst den
Aufbau und die Lichtausbreitung zusammen. Die Spalte mit Brechungsindex als
Überschrift stellt den Verlauf dieses grafisch dar. Beim Stufenindexprofil
steigen die Brechzahlen beim Übergang vom Mantel zum Kern abrupt an. Dies hat
die schon erwähnte Totalreflexion an der Kerngrenze, welche man in der Spalte
Querschnitt erkennen kann, zur Folge.

\begin{figure}[h]
    \begin{center}
        \begin{minipage}[t]{0.4\textwidth}
            \begin{center}
                \includegraphics[width=0.9\textwidth]{Bilder/Optische_Wellenleiter_Die_Polymer_Optische_Faser/Brechzahlprofile/pofsi.png}
                \caption[Aufbau des Stufenindexprofils \newline \url{http://www.itwissen.info/bilder/aufbau-und-brechungsprofil-der-stufenindex-profilfaser.png} (zuletzt aufgerufen am 19.09.2015)]{Aufbau des Stufenindexprofils}
                \label{fig:pofsi}
            \end{center}
        \end{minipage}
        \hspace{0.025\textwidth}
        \begin{minipage}[t]{0.4\textwidth}
            \begin{center}
                \includegraphics[height=0.1\textheight]{Bilder/Optische_Wellenleiter_Die_Polymer_Optische_Faser/Funktionsweise/pofdaempfung.png}
                \caption[Dämpfungsfenster bei einer polymer optischen Faser \newline \url{http://www.pofac.fh-nuernberg.de/pofac/de/was_sind_pof/images/pmma_daempfung.png} (zuletzt aufgerufen am 19.09.2015)]{Dämpfungsfenster beim Stufenprofil}
                \label{fig:pofdaempfung}
            \end{center}
        \end{minipage}
    \end{center}
\end{figure}

\autoref{fig:pofdaempfung} zeigt die Dämpfungsfenster einer polymer optischen
Faser. Diese liegen bei den Farben grün, gelb und rot. Um die Intensitätsabnahme
möglichst gering zu halten und damit die Reichweite zu erhöhen werden
Wellenlängen für die Lichtimpulse gewählt, die in den Dämpfungsfenstern liegen.
Als Lichtquelle kann zum Beispiel eine LED verwendet werden und als Empfänger
kommen Photodetektoren zum Einsatz. Der geringe Preis und die robuste
Übertragung auf kurzen Strecken machen die SI-POF zu einer beliebten alternative
gegenüber von Kupferkabeln in Industrieanlagen oder in Fahrzeugen. \cite{poflee}

\subsubsection{Gradientenindex}

Das Gradientenindexprofil bittet mit bis zu 10 Gigabit/s auf 100m deutlich
höhere Bitraten als das Stufenindexprofil. Beim Gradientenindex nimmt der
Brechungsindex vom Mantel bis zur Mitte des Kern kontinuierlich zu (siehe
\autoref{fig:pofgi} Spalte Brechungsindex). \autoref{fig:pofgi} zeigt ebenfalls
in der Spalte Längsschnitt den daraus resultierenden Lichtstrahlenverlauf.
Dieser verläuft, im Gegensatz zu der geraden Lichtausbreitung beim Stufenindex,
sinusförmig. Dadurch werden geringere Laufzeitunterschiede ermöglicht und die
Frequenz der Lichtimpulse kann erhöht werden. Außerdem wird bei der
Gradientenfaser eine Dämpfung von unter 20 dB/km erreicht (siehe
\autoref{fig:pofgidaempfung}. Dies ist ein ehrheblicher Sprung über der
Dämpfung einer Stufenfaser mit ca 100 dB/km (siehe \autoref{fig:pofdaempfung}).
Aus den beiden Gründen ist die Bandbreite einer Faser mit Gradientenindex
significant höher als die einer einer Faser mit Stufenindex. Aufgrund der hohen
Bitraten werden Gradientenfasern im \shorthandoff{"}"Local Area
Network"\shorthandon{"} (LAN) eingesetzt. Als Kernmaterial wird hier der
Kunststoff CYTOP\textsuperscript{\textregistered} der Asahi Glass Company
verwendet (siehe \autoref{subsec:pofcytop}). \cite{pofacgif}

\begin{figure}[h]
    \begin{center}
        \begin{minipage}[t]{0.4\textwidth}
            \begin{center}
                \includegraphics[height=0.1\textheight]{Bilder/Optische_Wellenleiter_Die_Polymer_Optische_Faser/Brechzahlprofile/pofgi.png}
                \caption[Aufbau des Gradientenindexprofils \newline \url{ITWissen}]{Aufbau des Gradientenindexprofils}
                \label{fig:pofgi}
            \end{center}
        \end{minipage}
        \hspace{0.025\textwidth}
        \begin{minipage}[t]{0.4\textwidth}
            \begin{center}
                \includegraphics[height=0.1\textheight]{Bilder/Optische_Wellenleiter_Die_Polymer_Optische_Faser/Brechzahlprofile/pofgidaempfung.png}
                \caption[Dämpfung bei einer Gradientenfaser \newline \url{POFAC}]{Dämpfung bei einer Gradientenfaser}
                \label{fig:pofgidaempfung}
            \end{center}
        \end{minipage}
    \end{center}
\end{figure}

%TODO: Get url


\subsubsection{Weitere Profile}

Weitere Erhöhungen der Bandbreite werden durch mehrere Mäntel (siehe
\autoref{fig:pofdsi} \autoref{fig:pofmsi}, Reduzierung der Laufzeitunterschiede)
bzw. durch mehrere Kerne (siehe \autoref{fig:pofmc}, mehrere Lichtstrahlen
gleichzeitig) innerhalb eines Kabels erreicht. \autoref{fig:pofdsimc} zeigt eine
Kombination der beiden obigen Optimierungsmöglichkeiten.
\cite{pofacprofile}

\begin{figure}[h]
    \begin{center}
        \begin{minipage}[t]{0.4\textwidth}
            \begin{center}
                \includegraphics[height=0.1\textheight]{Bilder/Optische_Wellenleiter_Die_Polymer_Optische_Faser/Brechzahlprofile/pofdsi.png}
                \caption[\shorthandoff{"}"\shorthandon{"}\shorthandoff{"}Dual
                Step Index"\shorthandon{"} - POF (zwei Mäntel, Stufenindex)
                \newline
                \url{POFAC}]{\shorthandoff{"}"\shorthandon{"}\shorthandoff{"}Dual
                Step Index"\shorthandon{"} - POF (zwei Mäntel, Stufenindex)}
                \label{fig:pofdsi}
            \end{center}
        \end{minipage}
        \hspace{0.025\textwidth}
        \begin{minipage}[t]{0.4\textwidth}
            \begin{center}
                \includegraphics[height=0.1\textheight]{Bilder/Optische_Wellenleiter_Die_Polymer_Optische_Faser/Brechzahlprofile/pofmsi.png}
                \caption[\shorthandoff{"}"\shorthandon{"}\shorthandoff{"}Multi
                Step Index"\shorthandon{"} - POF (mehrere Mäntel, Stufenindex)
                \newline
                \url{POFAC}]{\shorthandoff{"}"\shorthandon{"}\shorthandoff{"}Multi
                Step Index"\shorthandon{"} - POF (mehrere Mäntel, Stufenindex)}
                \label{fig:pofmsi}
            \end{center}
        \end{minipage}
    \end{center}

    \begin{center}
        \begin{minipage}[t]{0.4\textwidth}
            \begin{center}
                \includegraphics[height=0.1\textheight]{Bilder/Optische_Wellenleiter_Die_Polymer_Optische_Faser/Brechzahlprofile/pofmc.png}
                \caption[\shorthandoff{"}"\shorthandon{"}\shorthandoff{"}Multi
                Core"\shorthandon{"} - POF (mehrere Kerne, Stufenindex) \newline
                \url{POFAC}]{\shorthandoff{"}"\shorthandon{"}\shorthandoff{"}Multi
                Core"\shorthandon{"} - POF (mehrere Kerne, Stufenindex)}
                \label{fig:pofmc}
            \end{center}
        \end{minipage}
        \hspace{0.025\textwidth}
        \begin{minipage}[t]{0.4\textwidth}
            \begin{center}
                \includegraphics[height=0.1\textheight]{Bilder/Optische_Wellenleiter_Die_Polymer_Optische_Faser/Brechzahlprofile/pofdsimc.png}
                \caption[\shorthandoff{"}"\shorthandon{"}\shorthandoff{"}Dual
                Step Index - Multi Core"\shorthandon{"} - POF (mehrere Kerne,
                zwei Mäntel, Stufenindex) \newline
                \url{POFAC}]{\shorthandoff{"}"\shorthandon{"}\shorthandoff{"}Dual
                Step Index - Multi Core"\shorthandon{"} - POF (mehrere Kerne,
                zwei Mäntel, Stufenindex)}
                \label{fig:pofdsimc}
            \end{center}
        \end{minipage}
    \end{center}
\end{figure}
