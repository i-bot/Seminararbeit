\subsubsection{Polymerisation von Polymethylmethacrylat}
\label{subsec:pofpmma}

Polymethylmethacrylat (PMMA) wurde erstmals von Otto Röhm 1928 hergestellt. Fünf
Jahre später wurde PMMA unter dem Namen
Plexiglas\textsuperscript{\textregistered} vermarktet. Die hohe optisch
Durchlässigkeit und das geringe Gewicht eignen PMMA für den Einsatz in der
Automobielindustrie. Gerade diese Eigenschaften sind ebenfalls ein Grund für die
Verwendung in POFs. \cite{pofwuppmma}

Die Synthese von PMMA geschieht mittels einer radikalischen Polymerisation.
Dabei wird zu dem Monomer Methylmethacrylat (MMA) ein Radikalbildner
dazugegeben. Der Radikalbildner zerfällt durch Licht- / Wärmezufuhr in Radikale.
Dieses lagert sich an die C-C-Doppelbindung des Monomers an und gibt durch
Aufspaltung der Doppelbindung sein aktives Zentrum weiter. Dieser Vorgang
erzeugt wieder ein Radikale welches ebenfalls ein Methylmethacrylat angreift.
Dadurch wächst die Monomerkette zu einem Polymer an. Der Abbruch des
Kettenwachstums kann entweder durch einen Zusammenschluss mit einem weiteren
Radikal oder durch eine Disproportionierung geschehen. Beide Reaktionen haben
eine Sättigung des aktiven Zentrums zur Folge. \autoref{rec:pmma} fasst den
Reaktionsablauf zusammen und zeigt, dass es sich bei Polymethylmethacrylat um
ein lineares Makromolekül also um einen Thermoplasten handelt.

\begin{figure}[H]
    \begin{center}
        \footnotesize
        \setatomsep{1.7em}

        \chemnameinit{\chemfig{-[@{op,0.5}]CH_2-C(-[2]CH_3)(-[6]C(=[:-150]\lewis{36,O})(-[:-30]\lewis{57,O}-[:30]CH_3))-[@{cl,0.9},3.1pt]}}

        \chemname{\chemfig{R-R}}{\\\\Radikalbildner}
        \chemsign{+ $\scriptstyle n$}
        \chemname{\chemfig{H_2C=C(-[:60]CH_3)(-[:-60]C(=[:-120]\lewis{46,O})(-\lewis{26,O}-CH_3))}}{\\\\Methylmethacrylat}
        \chemrel{->}
        \chemname{\chemfig{-[@{op,0.5}]CH_2-C(-[2]CH_3)(-[6]C(=[:-150]\lewis{36,O})(-[:-30]\lewis{57,O}-[:30]CH_3))-[@{cl,0.9},3.1pt]}}{\\\\Polymethylmethacrylat}
        \makebraces[20pt,35pt]{n}{op}{cl}

        \caption{radikalische Polymerisation von Polymethylmethacrylat}
        \label{rec:pmma}
    \end{center}
\end{figure}
