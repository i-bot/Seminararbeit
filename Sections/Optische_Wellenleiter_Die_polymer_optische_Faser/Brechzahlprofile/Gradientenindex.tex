\subsubsection{Gradientenindexprofil}

Bei polymer optischen Fasern mit Gradientenindexprofil (GI-POF) nimmt die
Brechzahl vom Mantel bis zur Mitte des Kerns kontinuierlich zu (linke Grafik in
\autoref{fig:pofgi}), was zu einer sinusförmigen Lichtausbreitung in der Faser
führt (rechte Grafik in \autoref{fig:pofgi}).

Als Kernmaterial werden der Kunststoff CYTOP\textsuperscript{\texttrademark} der
Asahi Glass Company sowie Polymethylmethacrylat verwendet \cite{pofacgif}. Der
Kerndurchmesser einer GI-POF beträgt ca. 100 µm und liegt damit bei einem
Zehntel des Kerndurchmessers einer SI-POF. Der geringere Kerndurchmesser hat
geringere Laufzeitunterschiede zur Folge, was eine höhere Frequenz der
Lichtimpulse ermöglicht. Die Dämpfung einer GI-POF liegt unter 20 dB/km, was
gegenüber der SI-POF, deren Dämpfung bei ca. 100 dB/km (siehe
\autoref{fig:pofgidaempfung}) liegt, eine erhebliche Verbesserung darstellt.

Die Bandbreite von Fasern mit Gradientenindexprofil ist daher signifikant höher
als bei Fasern mit Stufenindexprofil. Es werden Durchsatzraten bis zu 40
Gigabit/s auf 100 m erreicht.

Aufgrund der hohen Bitraten und ihrer Biegsamkeit werden GI-POF vor allem in
\glqq Local Area Networks\grqq{} (LAN) und in Supercomputern eingesetzt.
\cite{poflee}


\begin{figure}[h]
    \begin{center}
        \begin{minipage}[t]{0.4\textwidth}
            \begin{center}
                \includegraphics[width=0.9\textwidth]{Bilder/Optische_Wellenleiter_Die_Polymer_Optische_Faser/Brechzahlprofile/pofgi.png}
                \caption[Aufbau des Gradientenindexprofils \newline \url{http://www.itwissen.info/bilder/aufbau-und-brechungsprofil-der-gradientenfaser.png} (zuletzt aufgerufen am 19.09.2015)]{Aufbau des Gradientenindexprofils}
                \label{fig:pofgi}
            \end{center}
        \end{minipage}
        \hspace{0.025\textwidth}
        \begin{minipage}[t]{0.4\textwidth}
            \begin{center}
                \includegraphics[height=0.1\textheight]{Bilder/Optische_Wellenleiter_Die_Polymer_Optische_Faser/Brechzahlprofile/pofgidaempfung.png}
                \caption[Dämpfung bei einer GI-POF \newline \url{http://www.pofac.fh-nuernberg.de/pofac/de/was_sind_pof/images/gradientenindex_daempfung.png} (zuletzt aufgerufen am 19.09.2015)]{Vergleich der Dämpfungswerte von GI-POF und SI-POF}
                \label{fig:pofgidaempfung}
            \end{center}
        \end{minipage}
    \end{center}
\end{figure}
