\section{Speicherung und Weitergabe von Informationen im historischen Kontext}

Informationen speichern und weitergeben ist zentraler Bestandteil der menschlichen Kultur und führte 
in der Menschheitsgeschichte zu bahnbrechenden Erfindungen und laufenden Innovationen.

In der Steinzeit wurden Informationen an Höhlenwände gemalt, ab dem 3. Jahrtausend vor Christus
mittels Keilschrift in Steintafeln gemeißelt und in der Antike auf Papyrus gezeichnet. Mitte
des 15. Jahrhunderts revolutionierte dann Gutenberg die Informationsweitergabe mit der Erfindung des
Buchdrucks, der die Produktion von Büchern in hohen Stückzahlen und zu geringen Kosten möglich
machte. Auch heute noch wird diese Technik für die Reproduktion von Texten verwendet.

\paragraph{}
Ende des 19. Jahrhunderts gelang es erstmals, Musik zu speichern. Hierfür wurden zunächst
Schallplatten aus Hartgummi eingesetzt. Dieser wurde um 1900 dann durch eine Pressmasse abgelöst
wurde, die im Wesentlichen aus Schellack bestand. Auf 12 Zoll-Schellackplatten konnten Musikstücke
mit ca. 4 Minuten Spielzeit pro Seite gespeichert werden. Mit dem Einsatz des Kunststoffes
Polyvinylchlorid ab den 50er Jahren des 20. Jahrhunderts verbesserte sich die Tonqualität von
Schallplatten deutlich. Außerdem konnte längere Stücke gespeichert und wiedergegeben werden.
\cite{schallplatte1}

Mit der Compact Cassette (CC), die 1963 auf den Markt kam, war es dann jedermann möglich, selbst
Musik aufzunehmen und dauerhaft zu speichern. Compact Cassetten verwenden ein Magnetband, das aus
einer langen schmalen Kunststofffolie besteht, die mit einem magnetisierbaren Material beschichtet
wurde. \cite{kassette1} \cite{kassette2}

In den 90er Jahren verdrängte  die Compact Disc (CD) innerhalb weniger Jahren sowohl die
Schallplatte als auch die Compact Cassette. Die CD zeichnet sich durch ihre hohe Speicherkapazität
sowie eine geringe Fehlerquote aus und wurde deshalb zum universellen Speichermedium für Musik, Dokumente,
Bilder und Filme. \cite{cd_durchbruch}

\paragraph{}
Kunststoffe sind - wie die oben aufgeführten Beispielen zeigen - für die Informationstechnologie
schon jetzt unersetzbar und kommen - neben der Speicherung - auch bei der Übertragung großer
Datenmengen zum Einsatz.

Sowohl für die Übertragung als auch für das Auslesen von Informationen werden Lichtsignale
verwendet. Hierfür werden vermehrt transparente Materialien benötigt, die sich für den
Alltagsgebrauch eignen.

\paragraph{}
An den Beispielen der Compact Disc und der polymer optischen Faser (POF) werden in dieser Arbeit
der Einsatz von transparenten Kunststoffen in der Informationstechnologie und ihre spezielle
Eignung behandelt. Dabei wird auf die Funktionsweise und Produktion von optischen Datenträgern und
optischen Wellenleitern eingangen. Außerdem werden die physikalischen Eigenschaften und die
Herstellung der verwendeten Kunststoffe erläutert.
